\title{Math 468 Homework 5}
\author{Ethan Balcik}
\date{\today}

\documentclass[12pt]{article}

\usepackage{tikz}
\usepackage{changepage}
\usepackage{amsmath}
\usepackage{amsfonts}
\usepackage{amsthm}
\usepackage{amssymb}
\usepackage{multirow}
\usepackage{hyperref}

\begin{document}
\maketitle

\noindent \textbf{Exercise 1.} Show that Reed-Muller codes have the following properties:\\
\begin{adjustwidth}{0.5cm}{}
\textbf{a.} $\mathcal{R}(i, m) \subset \mathcal{R}(j, m)$ for all $0 \leq i \leq j \leq m$.\\
	\begin{adjustwidth}{0.5cm}{}
	A basis for the vector space $\mathcal{V}$, which is the space of functions $f : \mathbb{F}_2^m \rightarrow \mathbb{F}_2$, may be given as,\\\\
	\centerline{$\mathcal{B} = \{0, 1, u_1, u_2, \dots , u_m , u_1 u_2, u_1 u_3, \dots , u_{m-1} u_{m}, \dots , u_1 u_2 \dots u_m\}$.}\\\\
	Furthermore, the Reed-Muller code $\mathcal{R}(i, m)$ may be thought of as the set generated by linearly combining any choice of $i$-degree or lower polynomials in $\mathcal{B}$.  Now consider $\mathcal{R}(j, m)$ where $j \geq i$.  Then $\mathcal{R}(j, m)$ must contain all linear combinations which compose $\mathcal{R}(i, m)$ since $\mathcal{R}(j, m)$ contains all linear combinations of $j$-degree or lower polynomials in $\mathcal{B}$ and $j \geq i$.\\
	\end{adjustwidth}
\textbf{b.} $dim(\mathcal{R}(r, m)) = \sum\limits_{i = 0}^{r} {m \choose i}$.\\
	\begin{adjustwidth}{0.5cm}{}
	A basis for the vector space $\mathcal{V}$, which is the space of functions $f : \mathbb{F}_2^m \rightarrow \mathbb{F}_2$, may be given as,\\\\
	\centerline{$\mathcal{B} = \{0, 1, u_1, u_2, \dots , u_m , u_1 u_2, u_1 u_3, \dots , u_{m-1} u_{m}, \dots , u_1 u_2 \dots u_m\}$.}\\\\
	Furthermore, the Reed-Muller code $\mathcal{R}(r, m)$ may be thought of as the set generated by linearly combining any choice of $r$-degree or lower polynomials in $\mathcal{B}$.  Therefore, a basis for $\mathcal{R}(r, m)$ is the set of all $r$-degree or lower polynomials in $\mathcal{B}$.  The dimension $dim(\mathcal{R}(r, m))$ is exactly the size of this basis.  Noting that there are exactly ${m \choose i}$ polynomials of degree $i$ in $\mathcal{B}$, we may deduce that $dim(\mathcal{R}(r, m)) = \sum\limits_{i = 0}^{r} {m \choose i}$ since the basis for $\mathcal{R}(r, m)$ is the set of all $r$-degree or lower polynomials in $\mathcal{B}$.\\
	\end{adjustwidth}
\textbf{c.} The minimum weight of $\mathcal{R}(r, m)$ is $2^{m - r}$.\\
	\begin{adjustwidth}{0.5cm}{}
	Consider the fact that $\mathcal{R}(0, 1)$ is just the repetition code of length $2$, and that $\mathcal{R}(1, 1)$ is just the entire vector space $\mathbb{F}_2^2$.  $\mathcal{R}(0, 1) = \{00, 11\}$, and thus clearly its minimum distance is $2 = 2^{1 - 0}$.  $\mathcal{R}(1, 1) = \{00, 01, 10, 11\}$, and again, clearly its minimum distance is $1 = 2^{1 - 1}$.  In fact, each of these cases are clearly true for arbitrary $m$ since the repetition code of length $m$ has minimum distance $m$, and since the minimum distance of any vector space in its entirety is simply $1$.\\\\
	Now assume that the minimum distance of $\mathcal{R}(r, m)$ is given as $2^{m - r}$ for all cases $\leq m$.  Thus, the minimum distance of $\mathcal{R}(r-1, m)$ is $2^{m - r + 1}$ by assumption.  Consider the $(u, u+v)$ construction of $\mathcal{R}(r - 1, m)$ and $\mathcal{R}(r, m)$, which is $\mathcal{R}(r, m+1)$.  Then the minimum distance of $\mathcal{R}(r, m+1)$ is,\\\\
	\centerline{$min(2(2^{m - r}), 2^{m - r + 1}) = min(2^{m - r + 1}, 2^{m - r + 1}) = 2^{m - r + 1} = 2^{(m + 1) - r}$}\\\\
	The result follows by induction.\\
	\end{adjustwidth}
\textbf{d.} $\mathcal{R}(m, m)^{\bot} = \{0\}$ and for all $0 \leq r < m$, the dual of $\mathcal{R}(r, m)$ is $\mathcal{R}(m-r-1, m)$.\\
	\begin{adjustwidth}{0.5cm}{}
	Recall that $\mathcal{R}(m, m)$ is just the whole vector space $\mathbb{F}_2^{2^m}$.  We note that $\{0\} \subseteq \mathcal{R}(m, m)^{\bot}$ since the inner product of the zero vector and any other vector in a vector space is $0$.  Now suppose we add any nonzero vector to the set $\{0\}$ with weight $0 < w \leq 2^m$.  Then we may find a vector $\vec{v} \in \mathcal{R}(m, m)$ which differs from $w - 1$ of this vector's nonzero coordinates since $\mathcal{R}(m, m)$ is just the whole vector space $\mathbb{F}_2^{2^m}$.  Thus, we may not add any nonzero vector to the set $\{0\}$ while still maintaining its "dualness" to $\mathcal{R}(m, m)$.  Thus, $\mathcal{R}(m, m)^{\bot} = \{0\}$.\\\\
	Next, note that,\\\\
	\centerline{$\sum\limits_{i = 0}^{r} {m \choose i} + \sum\limits_{i = 0}^{m-r-1} {m \choose i} = \sum\limits_{i = 0}^{r} {m \choose i} + \sum\limits_{i = (r + 1)}^{m} {m \choose i} = \sum\limits_{i = 0}^{m} {m \choose i} = 2^m$.}\\\\
	We now need only to introduce further structure on the vector space $\mathbb{F}_2^{2^m}$ in the form of Affine Geometry in order to show that $\mathcal{R}(r, m)^{\bot} = \mathcal{R}(m-r-1, m)$.  We simply need to show that the wedge product of a basis vector $u$ of $\mathcal{R}(r, m)$ and a basis vector $v$ of $\mathcal{R}(m-r-1, m)$, has even weight, and this follows due to the property of Affine Geometry which dictates that the characteristic function of some k-flat has weight $2^{m-k}$.\\
	\end{adjustwidth}
\textbf{e.} $\mathcal{R}(m-2, m)$ are extended Hamming codes of length $2^m$.\\
	\begin{adjustwidth}{0.5cm}{}
	The parity check matrix for the extended Hamming codes of length $2^m$ has a recursive structure which can be described as follows.  Label the first $2^{m-1}$ entries of the first row in the parity check matrix $0$, and the rest $1$.  For the second row, choose the first $2^{m-2}$ values from either group of $2^{m-1}$ entries to form the first group of $2^{m-1}$ entries in the second row, and repeat for the remaining entries from the first row to form the second group of $2^{m-1}$ entries in the second row.  In general, for the $i$th row, choose the first $2^{m-i}$ entries from each group of $2^{m-i+1}$ entries from the previous row, and this forms the first group of $2^{m-1}$ entries in the $i$th row, which is repeated for the second group of $2^{m-1}$ entries in the $i$th row.  Finally, add a row containing entirely $1$s as its entries to complete the parity check matrix.  For example, consider the parity check matrix for the extended Hamming code of length $2^3$.\\\\
	\centerline{$
		\begin{bmatrix}
			1 & 1 & 1 & 1 & 1 & 1 & 1 & 1\\
			0 & 0 & 0 & 0 & 1 & 1 & 1 & 1\\
			0 & 0 & 1 & 1 & 0 & 0 & 1 & 1\\
			0 & 1 & 0 & 1 & 0 & 1 & 0 & 1
		\end{bmatrix}
	$}\\\\
	Matrices of this form are exactly the generator matrices for $\mathcal{R}(1, m)$.  Noting that the parity check matrix is the generator matrix of the dual code, we must simply show that $\mathcal{R}(1, m)^{\bot} = \mathcal{R}(m-2, m)$.  By the result in part \textbf{d}, we have that $\mathcal{R}(r, m)^{\bot} = \mathcal{R}(m-r-1, m)$.  Thus, $\mathcal{R}(1, m)^{\bot} = \mathcal{R}(m-1-1, m) = \mathcal{R}(m-2, m)$.\\
	\end{adjustwidth}
\textbf{f.} $\mathcal{R}(1, m)$ consists of the rows of the Hadamard matrix $H_{2^m} = H_2 \otimes \dots \otimes H_2$, where we change the $1$ to $0$ and $-1$ to $1$, together with their complements.\\
	\begin{adjustwidth}{0.5cm}{}
	Consider $\mathcal{R}(1, 1) = \{00, 01, 10, 11\}$.  The Hadamard matrix $H_{2}$ with its entries replaced according to the above rule is given as,\\\\
	\centerline{$
		\begin{bmatrix}
			0 & 0\\
			0 & 1
		\end{bmatrix}
	$.}\\\\
	Thus, the rows of this matrix together with their complements form the set $\{00, 01, 10, 11\}$, which is exactly $\mathcal{R}(1, 1)$.  Next consider $\mathcal{R}(1, 2) = \{0000, 0011, 0101, 0110, 1001, 1010, 1100, 1111\}$.  The Hadamard matrix $H_{4}$ with its entries replaced according to the above rule is given as,\\\\
	\centerline{$
		\begin{bmatrix}
			0 & 0 & 0 & 0\\
			0 & 1 & 0 & 1\\
			0 & 0 & 1 & 1\\
			0 & 1 & 1 & 0
		\end{bmatrix}
	$.}\\\\
	Thus, the rows of this matrix together with their complements form the set $\{0000, 0011, 0101, 0110, 1001, 1010, 1100, 1111\}$, which is exactly $\mathcal{R}(1, 2)$.\\\\
	Interestingly, if we treat the rows of $H_2$ as the code words of some code, we notice that the following Hadamard matrix (in this case $H_4$) is the $(u, u+v)$ construction of $H_2$ and the repetition code of length $2$.  Then, together with its complements, it corresponds exactly to the $(u, u+v)$ construction of $\mathcal{R}(1, 1)$ and $\mathcal{R}(0, 1)$.  This observation motivates the following proof by induction.\\\\
	Assume that the rows of $H_{2^m}$ along with their complements forms $\mathcal{R}(1, m)$ for all cases $\leq m$.  Then consider the hadamard matrix $H_{2^{(m+1)}}$.  By the previous observation, this matrix corresponds to the $(u, u+v)$ construction of itself and the repetition code, if we think of this matrix as representing code words of some code.  Then, together with its complements, this corresponds exactly to the $(u, u+v)$ construction of $\mathcal{R}(0, m)$ and $\mathcal{R}(1, m)$, which yields $\mathcal{R}(1, m+1)$.  The result follows by induction.\\
	\end{adjustwidth}
\end{adjustwidth}

\noindent \textbf{Exercise 2.} Show that the $(u, u+v)$-construction with $C_1 = \mathcal{R}(r+1, m)$, $C_2 = \mathcal{R}(r, m)$ yields $C = \mathcal{R}(r+1, m+1)$.\\
\begin{adjustwidth}{0.5cm}{}
We may show this by first noting that the Reed-Muller code $\mathcal{R}(r, m)$ may be constructed recursively using $(u, u+v)$ construction in the following manner.  Let the base cases be denoted $\mathcal{R}(0, m)$, being the repetition code, and $\mathcal{R}(m, m)$, being the entire vector space $\mathbb{F}_2^{2^m}$.  From some number of base cases, one may construct a Reed-Muller code using $(u, u+v)$ construction with $\mathcal{R}(r, m)$ being the $(u, u+v)$ construction of $\mathcal{R}(r, m-1)$ and $\mathcal{R}(r-1, m-1)$.\\\\
By this information, we may deduce that, if $C = \mathcal{R}(r+1, m+1)$, then $C$ is the $(u, u+v)$ construction of\\\\
	\centerline{$C_1 = \mathcal{R}(r+1, (m+1)-1)$}\\\\
	\centerline{$\Rightarrow C_1 = \mathcal{R}(r+1, m)$}\\\\
and,\\\\
	\centerline{$C_2 = \mathcal{R}((r+1)-1, (m+1)-1)$}\\\\
	\centerline{$\Rightarrow C_2 = \mathcal{R}(r, m)$}\\
\end{adjustwidth}

\noindent \textbf{Exercise 3.} Compute the weight enumerator of the Golay code $[23, 12, 7]$.\\
\begin{adjustwidth}{0.5cm}{}
We show that the weight enumerator of the Golay code is\\\\
	\centerline{$1 + 253x^7 + 506x^8 + 1288x^{11} + 1288x^{12} + 506x^{15} + 253x^{16} + x^{23}$}\\\\
by calculating the weight distribution by brute-force calculation.  Please see \href{https://github.com/whatsacomputertho/math-468-homework-5}{the repository} for the source code used to show this.\\
\end{adjustwidth}

\noindent \textbf{Exercise 4.} Show that the extended Golay code $[24, 12, 8]$ is self-dual.\\
\begin{adjustwidth}{0.5cm}{}
We show that the extended Golay code $[24, 12, 8]$ is self-dual by use of brute-force calculation on its generator matrix.  Please see \href{https://github.com/whatsacomputertho/math-468-homework-5}{the repository} for the source code used to show this.\\
\end{adjustwidth}

\end{document}